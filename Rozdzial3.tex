\chapter{Analiza wymagań i projekt rozwiązania}
Rozdział przedstawia wymagania oraz projekt rozwiązania, zgodny z~charakterem pracy
inżynierskiej o~profilu projektowym. Opis obejmuje zarówno aspekty funkcjonalne,
jak i niefunkcjonalne, a~także strukturę systemu i proces kompilacji dokumentów.

\section{Wymagania funkcjonalne}
Do kluczowych wymagań funkcjonalnych należą:
\begin{itemize}
    \item możliwość kompilacji dokumentu \LaTeX{} do postaci PDF w~środowisku kontenerowym,
    \item automatyzacja wielokrotnego przebiegu kompilacji (np. dla spisu treści),
    \item czyszczenie plików tymczasowych po zakończeniu procesu,
    \item czytelna informacja o~wyniku kompilacji dla użytkownika.
\end{itemize}

\section{Wymagania niefunkcjonalne}
Rozwiązanie powinno spełniać następujące wymagania jakościowe:
\begin{itemize}
    \item przenośność pomiędzy systemami operacyjnymi,
    \item powtarzalność wyników kompilacji niezależnie od środowiska hosta,
    \item łatwość uruchomienia przez użytkownika nietechnicznego,
    \item bezpieczeństwo wynikające z~izolacji procesu kompilacji.
\end{itemize}

\section{Architektura rozwiązania}
Architektura przyjmuje układ warstwowy: warstwa dokumentu (źródła \LaTeX{}),
warstwa automatyzacji (skrypt uruchomieniowy) oraz warstwa środowiska uruchomieniowego
(kontener Docker z~zainstalowanym \LaTeX{}). Taki podział umożliwia niezależne
zarządzanie treścią dokumentu i procesem budowania.

\section{Projekt procesu kompilacji}
Proces kompilacji obejmuje:
\begin{itemize}
    \item przygotowanie katalogu roboczego i usunięcie plików pomocniczych,
    \item uruchomienie kontenera i wykonanie kompilacji \texttt{latexmk} w~trybie PDF,
    \item ponowny przebieg kompilacji w~celu odświeżenia spisów,
    \item weryfikację istnienia pliku wynikowego i przekazanie komunikatu użytkownikowi.
\end{itemize}

\section{Struktura projektu}
Projekt składa się z~pliku głównego dokumentu oraz rozdziałów tematycznych,
uzupełnionych o~pliki bibliografii i zasoby graficzne. Struktura jest zgodna
z~przyjętymi standardami prac inżynierskich i ułatwia dalszą rozbudowę.