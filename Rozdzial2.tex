\chapter{Problematyka projektu}
Rozdział przedstawia problem projektowy związany z~przygotowaniem dokumentów w~\LaTeX{}
oraz trudnościami w~zapewnieniu powtarzalnego procesu kompilacji na różnych stanowiskach.
Kluczowym wyzwaniem jest ujednolicenie środowiska narzędziowego oraz uproszczenie procesu
tworzenia pliku wynikowego dla użytkownika końcowego.

\section{Opis problemu}
W~praktyce akademickiej często występują rozbieżności w~konfiguracji narzędzi \LaTeX{},
co skutkuje błędami kompilacji, różnicami w~wyniku składu lub brakiem możliwości
odtworzenia wyników na innym komputerze. Problemem jest również złożoność procesu
kompilacji, wymagającego znajomości wielu parametrów i narzędzi pomocniczych.
Projekt ma na celu uproszczenie tego procesu oraz zapewnienie powtarzalności wyników.

\section{Cel i zakres pracy}
Głównym celem projektu jest przygotowanie rozwiązania, które umożliwia kompilację
dokumentów \LaTeX{} w~izolowanym środowisku kontenerowym oraz dostarcza prosty mechanizm
uruchomienia procesu kompilacji. Zakres obejmuje analizę wymagań, projekt architektury,
implementację i weryfikację działania rozwiązania.

\section{Uzasadnienie wyboru tematu}
Wybór tematu wynika z~praktycznych potrzeb studentów i promotorów w~zakresie tworzenia
prac dyplomowych oraz rosnącej popularności narzędzi kontenerowych w~inżynierii oprogramowania.
W~literaturze podkreśla się znaczenie automatyzacji procesu składu i jego niezawodności
\parencite{goossens97}.

\section{Pytania badawcze}
W~ramach projektu sformułowano następujące pytania:
\begin{itemize}
	\item W~jaki sposób ujednolicić środowisko kompilacji dokumentów \LaTeX{} na różnych systemach?
	\item Jak zaprojektować proces kompilacji, aby był prosty w~użyciu i możliwy do automatyzacji?
	\item Jakie wymagania funkcjonalne i niefunkcjonalne są kluczowe dla użytkownika końcowego?
\end{itemize}