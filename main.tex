\documentclass[12pt, a4paper, oneside]{report}

% --- Pakiety podstawowe ---
\usepackage[utf8]{inputenc}
\usepackage[T1]{fontenc}
% --- Dodanie wąskiej czcionki programistycznej ---
\usepackage[varqu,varl]{zi4} % Inconsolata - smukła, nowoczesna i czytelna czcionka do kodu
\usepackage[polish]{babel}
\usepackage{lmodern, indentfirst, geometry, setspace, graphicx, xcolor}
\geometry{lmargin=3cm, rmargin=2.5cm, tmargin=2.5cm, bmargin=2.5cm}
\setlength{\parindent}{0.75cm}
\usepackage{chngcntr}

% --- POPRAWKA 1: Globalna interlinia ---
\onehalfspacing 

% --- Rysunki UML i TikZ ---
\usepackage{tikz}
\usetikzlibrary{arrows.meta, positioning, shapes}
% --- Ścieżka do rysunków ---
\usepackage{graphicx}
\graphicspath{ {./rysunki/} } 

% --- Formatowanie nagłówków ---
\usepackage{titlesec}
\titleformat{\chapter}[hang]{\normalfont\fontsize{16}{20}\selectfont\bfseries}{\thechapter.}{0.5em}{}
\titleformat{\section}[hang]{\normalfont\fontsize{14}{18}\selectfont\bfseries}{\thesection.}{0.5em}{}
\titleformat{\subsection}[hang]{\normalfont\fontsize{14}{18}\selectfont\bfseries}{\thesubsection.}{0.5em}{}

\titlespacing*{\chapter}{0pt}{12pt}{12pt}
\titlespacing*{\section}{0pt}{6pt}{6pt}
\titlespacing*{\subsection}{0pt}{6pt}{6pt}

% --- Konfiguracja podpisów ---
\usepackage{caption}
% Tabele: Tytuł nad tabelą, format "Tabela 1" <enter> "Tytuł" (bez kropki)
\captionsetup[table]{
    labelsep=newline,
    justification=raggedright,
    singlelinecheck=false,
    font={normalsize,normalfont},
    labelfont={normalsize,normalfont},
    skip=6pt
}
% Rysunki: Format "Rysunek X. Tytuł" (z kropką), 11pt, wyrównanie do lewej
\captionsetup[figure]{
    labelsep=period,
    justification=raggedright,
    singlelinecheck=false,
    font={small,normalfont},
    labelfont={small,normalfont},
    skip=6pt
}

% --- Numeracja i odstępy obiektów ---
\counterwithout{table}{chapter}
\counterwithout{figure}{chapter}
\setlength{\intextsep}{12pt plus 2pt minus 2pt}
\setlength{\textfloatsep}{15pt plus 2pt minus 2pt}
\setlength{\floatsep}{12pt plus 2pt minus 2pt}

% --- Bibliografia ---
\usepackage[backend=biber, style=authoryear, sorting=nyt]{biblatex}
\addbibresource{Bibliografia.bib}
\addbibresource{Netografia.bib}
\renewcommand*{\nameyeardelim}{\addcomma\space}

% --- Listingi ---
\usepackage{booktabs, listings}
\addto\captionspolish{\renewcommand{\lstlistingname}{Listing}}
% Usunięcie numeracji rozdziałowej dla listingów
\AtBeginDocument{\counterwithout{lstlisting}{chapter}}
% Konfiguracja podpisów listingów - 11pt
\captionsetup[lstlisting]{
    labelsep=period,
    justification=raggedright,
    singlelinecheck=false,
    font={small,normalfont},
    labelfont={small,normalfont},
    skip=6pt
}

% --- Listingi ---
\lstset{
    backgroundcolor=\color{gray!10}, 
    % \small i Inconsolata (\ttfamily) sprawią, że kod będzie bardzo zwarty
    basicstyle=\small\ttfamily\setstretch{0.9}, 
    columns=fullflexible,    % Najmocniejsze zwężenie odstępów między literami
    keepspaces=true, 
    breaklines=true, 
    tabsize=4,               % Standardowa szerokość tabulacji
    numbers=left, 
    numberstyle=\tiny\color{gray},
    keywordstyle=\color{blue}\bfseries, 
    commentstyle=\color{green!50!black},
    stringstyle=\color{orange},
    inputencoding=utf8, 
    extendedchars=true,
    literate={ą}{{\k{a}}}1 {ć}{{\'c}}1 {ę}{{\k{e}}}1 {ł}{{\l{}}}1 {ń}{{\'n}}1 {ó}{{\'o}}1 {ś}{{\'s}}1 {ź}{{\'z}}1 {ż}{{\.z}}1
}

% --- Spis treści (TOC) ---
\usepackage{tocloft}
\renewcommand{\cftchapleader}{\cftdotfill{\cftdotsep}}
\renewcommand{\cftchappagefont}{\normalfont}
\renewcommand{\cftchapfont}{\normalfont}
\setlength{\cftbeforechapskip}{0pt}

% --- Kropki po numerach w spisie treści ---
\renewcommand{\cftchapaftersnum}{.} 
\renewcommand{\cftsecaftersnum}{.}  
\renewcommand{\cftsubsecaftersnum}{.}

% ---  Ustawienie odstępu, aby numer z kropką nie nachodził na tekst ---
\addtolength{\cftchapnumwidth}{0pt}
\addtolength{\cftsecnumwidth}{0pt}
\addtolength{\cftsubsecnumwidth}{0pt}

% --- Linki ---
\usepackage{hyperref}
\hypersetup{hidelinks}

% =================================================================

% --- Numeracja stron na dole, na środku ---
\usepackage{fancyhdr}
\pagestyle{plain}
\fancyhf{}
\fancyfoot[C]{\thepage}
\renewcommand{\headrulewidth}{0pt}

\begin{document}

% ============================================================================
% 1.                            Strona tytułowa
% ============================================================================
\title{Nazwa przedmiotu (projekt)\\tu Tytuł Twojej Pracy}
\author{Imię Nazwisko \\ nr albumu: 123456 \\ kierunek: Informatyka \\ rok: 2025/2026}
\maketitle

\pagenumbering{arabic}
\setcounter{page}{1}
% ============================================================================

% 2. Spis treści
\newpage
% POPRAWKA 3: Usunięto zbędne środowisko \begin{spacing}
\tableofcontents
\clearpage

% 3. Treść merytoryczna
\chapter*{Wstęp}
\addcontentsline{toc}{chapter}{Wstęp}
% --- Wstęp do pracy ---
Wstęp powinien zawierać krótki opis tematu pracy, jej cel oraz zakres. Możesz również wspomnieć o strukturze dokumentu i metodach badawczych, które zostaną użyte.

%===============================================================================================
%--- Przykład cytowania i kodu źródłowego, tabel, obrazów i listingów ---
Poniżej znajduje się przykładowy fragment tekstu zawierający cytowanie 
oraz listing kodu źródłowego. Wzory tabel, grafik itp. 
Stanowią one jedynie wzór do dalszego rozwinięcia pracy.
%===============================================================================================
%--- Przykłady cytowania, ----
\section*{Przykład cytowania}

To jest przykład cytowania \cite{lamport94}. Poniżej kod:

%===============================================================================================
%--- Przykład kodu źródłowego w CPP ---
\section*{Przykład kodu źródłowego w CPP}
\begin{lstlisting}[language=Python, caption={Program sumujący dwie liczby w Pythonie}]
def main():
    liczba1 = float(input("Podaj pierwszą liczbę: "))
    liczba2 = float(input("Podaj drugą liczbę: "))
    suma = liczba1 + liczba2
    print(f"Suma: {suma}")

if __name__ == "__main__":
    main()
\end{lstlisting}
%--- Przykład kodu źródłowego w JAVA ---
\section*{Przykład kodu źródłowego w JAVA}
\begin{lstlisting}[language=Java, caption={Program sumujący dwie liczby w Javie}]
import java.util.Scanner;

public class Suma {
    public static void main(String[] args) {
        Scanner scanner = new Scanner(System.in);
        System.out.print("Podaj pierwszą liczbę: ");
        float liczba1 = scanner.nextFloat();
        System.out.print("Podaj drugą liczbę: ");
        float liczba2 = scanner.nextFloat();
        float suma = liczba1 + liczba2;
        System.out.println("Suma: " + suma);
        scanner.close();
    }
}
\end{lstlisting}

%--- Przykład kodu źródłowego w CPP ---
\section*{Przykład kodu źródłowego w CPP}
\begin{lstlisting}[language=C++, caption={Program sumujący dwie liczby w C++}]
#include <iostream>
using namespace std;

int main() {
    float liczba1, liczba2;
    cout << "Podaj pierwszą liczbę: ";
    cin >> liczba1;
    cout << "Podaj drugą liczbę: ";
    cin >> liczba2;
    float suma = liczba1 + liczba2;
    cout << "Suma: " << suma << endl;
    return 0;
}
\end{lstlisting}


%--- Przykład zapytania SQL ---
\section*{Przykład zapytania SQL}   
\begin{lstlisting}[language=SQL, caption={Zapytanie SQL wybierające wszystkie rekordy z tabeli 'uzytkownicy'}]
SELECT * FROM uzytkownicy;  
\end{lstlisting}

%===============================================================================================
%--- Przykład tabel ---
\section*{Przykład tabeli}  

\begin{table}[ht!]
\centering
\caption{Przykładowa tabela z trzema kolumnami sformatowana zgodnie z zasadami}
\label{tab:przyklad}
\begin{tabular}{lll}
\toprule
Nazwa elementu & Opis funkcjonalności & Wartość [j.] \\  
\midrule
Element A      & Opis szczegółowy A    & 10 \\
Element B      & Opis szczegółowy B    & 20 \\ 
Element C      & Opis szczegółowy C    & 30 \\
\bottomrule 
\end{tabular}
\sourcetab{opracowanie własne}
\end{table}

%===============================================================================================
%--- Przykład obrazu ---
\begin{figure}[ht]
    \centering
    \includegraphics[width=0.7\textwidth]{rysunki/1.png} 
    \caption{Dodatkowy schemat systemu (import z pliku PNG)}
    \label{fig:schemat_png} % Klucz dla grafiki PNG
\end{figure}

\noindent Na rysunku \ref{fig:schemat_png} zaprezentowano strukturę systemu w ujęciu ogólnym.

%===============================================================================================
%przykład odwołania do bibliografii
\section*{Przykład odwołania do bibliografii}
W literaturze przedmiotu \parencite{nowak2020} omówiono zagadnienia związane z...

%===============================================================================================
%przykład owodłania do netografii
\section*{Przykład odwołania do netografii}
W Internecie dostępne są zasoby dotyczące... \parencite{latex_project}
%===============================================================================================
\chapter{Temat rozdziału}
Tutaj treść rozdziału 1. Możesz dopisać swoje podrozdziały i sekcje tutaj. 
Pamiętaj, aby nie zostawiać pustej linii bezpośrednio przed otwarciem środowiska tabeli, jeśli chcesz, aby była ona bliżej tekstu.

\section{Podrozdział albo sekcja }
Przykładowy tekst, który prowadzi bezpośrednio do prezentacji danych. Warto wspomnieć o literaturze przedmiotu \parencite{kowalski2026}.

\begin{table}[ht!]
\centering
\caption{Przykładowa tabela z trzema kolumnami sformatowana zgodnie z zasadami}
\label{tab:przyklad}
\begin{tabular}{llr}
\toprule
Nazwa elementu & Opis funkcjonalności & Wartość [j.] \\  
\midrule
Element A      & Opis szczegółowy A    & 10 \\
Element B      & Opis szczegółowy B    & 20 \\ 
Element C      & Opis szczegółowy C    & 30 \\
\bottomrule 
\end{tabular}
\sourcetab{opracowanie własne}
\end{table}

Kolejny akapit tekstu zaczyna się tutaj. 
\chapter{Problematyka projektu}
Rozdział przedstawia problem projektowy związany z~przygotowaniem dokumentów w~\LaTeX{}
oraz trudnościami w~zapewnieniu powtarzalnego procesu kompilacji na różnych stanowiskach.
Kluczowym wyzwaniem jest ujednolicenie środowiska narzędziowego oraz uproszczenie procesu
tworzenia pliku wynikowego dla użytkownika końcowego.

\section{Opis problemu}
W~praktyce akademickiej często występują rozbieżności w~konfiguracji narzędzi \LaTeX{},
co skutkuje błędami kompilacji, różnicami w~wyniku składu lub brakiem możliwości
odtworzenia wyników na innym komputerze. Problemem jest również złożoność procesu
kompilacji, wymagającego znajomości wielu parametrów i narzędzi pomocniczych.
Projekt ma na celu uproszczenie tego procesu oraz zapewnienie powtarzalności wyników.

\section{Cel i zakres pracy}
Głównym celem projektu jest przygotowanie rozwiązania, które umożliwia kompilację
dokumentów \LaTeX{} w~izolowanym środowisku kontenerowym oraz dostarcza prosty mechanizm
uruchomienia procesu kompilacji. Zakres obejmuje analizę wymagań, projekt architektury,
implementację i weryfikację działania rozwiązania.

\section{Uzasadnienie wyboru tematu}
Wybór tematu wynika z~praktycznych potrzeb studentów i promotorów w~zakresie tworzenia
prac dyplomowych oraz rosnącej popularności narzędzi kontenerowych w~inżynierii oprogramowania.
W~literaturze podkreśla się znaczenie automatyzacji procesu składu i jego niezawodności
\parencite{goossens97}.

\section{Pytania badawcze}
W~ramach projektu sformułowano następujące pytania:
\begin{itemize}
	\item W~jaki sposób ujednolicić środowisko kompilacji dokumentów \LaTeX{} na różnych systemach?
	\item Jak zaprojektować proces kompilacji, aby był prosty w~użyciu i możliwy do automatyzacji?
	\item Jakie wymagania funkcjonalne i niefunkcjonalne są kluczowe dla użytkownika końcowego?
\end{itemize}
\chapter{Analiza wymagań i projekt rozwiązania}
Rozdział przedstawia wymagania oraz projekt rozwiązania, zgodny z~charakterem pracy
inżynierskiej o~profilu projektowym. Opis obejmuje zarówno aspekty funkcjonalne,
jak i niefunkcjonalne, a~także strukturę systemu i proces kompilacji dokumentów.

\section{Wymagania funkcjonalne}
Do kluczowych wymagań funkcjonalnych należą:
\begin{itemize}
    \item możliwość kompilacji dokumentu \LaTeX{} do postaci PDF w~środowisku kontenerowym,
    \item automatyzacja wielokrotnego przebiegu kompilacji (np. dla spisu treści),
    \item czyszczenie plików tymczasowych po zakończeniu procesu,
    \item czytelna informacja o~wyniku kompilacji dla użytkownika.
\end{itemize}

\section{Wymagania niefunkcjonalne}
Rozwiązanie powinno spełniać następujące wymagania jakościowe:
\begin{itemize}
    \item przenośność pomiędzy systemami operacyjnymi,
    \item powtarzalność wyników kompilacji niezależnie od środowiska hosta,
    \item łatwość uruchomienia przez użytkownika nietechnicznego,
    \item bezpieczeństwo wynikające z~izolacji procesu kompilacji.
\end{itemize}

\section{Architektura rozwiązania}
Architektura przyjmuje układ warstwowy: warstwa dokumentu (źródła \LaTeX{}),
warstwa automatyzacji (skrypt uruchomieniowy) oraz warstwa środowiska uruchomieniowego
(kontener Docker z~zainstalowanym \LaTeX{}). Taki podział umożliwia niezależne
zarządzanie treścią dokumentu i procesem budowania.

\section{Projekt procesu kompilacji}
Proces kompilacji obejmuje:
\begin{itemize}
    \item przygotowanie katalogu roboczego i usunięcie plików pomocniczych,
    \item uruchomienie kontenera i wykonanie kompilacji \texttt{latexmk} w~trybie PDF,
    \item ponowny przebieg kompilacji w~celu odświeżenia spisów,
    \item weryfikację istnienia pliku wynikowego i przekazanie komunikatu użytkownikowi.
\end{itemize}

\section{Struktura projektu}
Projekt składa się z~pliku głównego dokumentu oraz rozdziałów tematycznych,
uzupełnionych o~pliki bibliografii i zasoby graficzne. Struktura jest zgodna
z~przyjętymi standardami prac inżynierskich i ułatwia dalszą rozbudowę.
\chapter*{Podsumowanie}
\phantomsection
\addcontentsline{toc}{chapter}{Podsumowanie}

Tu znajduje się krótkie podsumowanie pracy — przedstaw główne wnioski, ograniczenia oraz propozycje dalszych badań.
...

% 4. Bibliografia
\clearpage
\phantomsection
\printbibliography[title={Bibliografia}, nottype=online, heading=bibintoc]

\clearpage
\phantomsection
\printbibliography[title={Netografia}, type=online, heading=bibintoc]

% 5. Spis rysunków
\clearpage
\phantomsection
\addcontentsline{toc}{chapter}{\listfigurename}
\listoffigures

% 6. Spis tabel
\clearpage
\phantomsection 
\addcontentsline{toc}{chapter}{\listtablename}
\listoftables

\end{document}