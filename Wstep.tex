\chapter*{Wstęp}
\addcontentsline{toc}{chapter}{Wstęp}
% --- Wstęp do pracy ---
Wstęp powinien zawierać krótki opis tematu pracy, jej cel oraz zakres. Możesz również wspomnieć o strukturze dokumentu i metodach badawczych, które zostaną użyte.

%===============================================================================================
%--- Przykład cytowania i kodu źródłowego, tabel, obrazów i listingów ---
Poniżej znajduje się przykładowy fragment tekstu zawierający cytowanie 
oraz listing kodu źródłowego. Wzory tabel, grafik itp. 
Stanowią one jedynie wzór do dalszego rozwinięcia pracy.
%===============================================================================================
%--- Przykłady cytowania, ----
\section*{Przykład cytowania}

To jest przykład cytowania \cite{lamport94}. Poniżej kod:

%===============================================================================================
%--- Przykład kodu źródłowego w CPP ---
\section*{Przykład kodu źródłowego w CPP}
\begin{lstlisting}[language=Python, caption={Program sumujący dwie liczby w Pythonie}]
def main():
    liczba1 = float(input("Podaj pierwszą liczbę: "))
    liczba2 = float(input("Podaj drugą liczbę: "))
    suma = liczba1 + liczba2
    print(f"Suma: {suma}")

if __name__ == "__main__":
    main()
\end{lstlisting}
%--- Przykład kodu źródłowego w JAVA ---
\section*{Przykład kodu źródłowego w JAVA}
\begin{lstlisting}[language=Java, caption={Program sumujący dwie liczby w Javie}]
import java.util.Scanner;

public class Suma {
    public static void main(String[] args) {
        Scanner scanner = new Scanner(System.in);
        System.out.print("Podaj pierwszą liczbę: ");
        float liczba1 = scanner.nextFloat();
        System.out.print("Podaj drugą liczbę: ");
        float liczba2 = scanner.nextFloat();
        float suma = liczba1 + liczba2;
        System.out.println("Suma: " + suma);
        scanner.close();
    }
}
\end{lstlisting}

%--- Przykład kodu źródłowego w CPP ---
\section*{Przykład kodu źródłowego w CPP}
\begin{lstlisting}[language=C++, caption={Program sumujący dwie liczby w C++}]
#include <iostream>
using namespace std;

int main() {
    float liczba1, liczba2;
    cout << "Podaj pierwszą liczbę: ";
    cin >> liczba1;
    cout << "Podaj drugą liczbę: ";
    cin >> liczba2;
    float suma = liczba1 + liczba2;
    cout << "Suma: " << suma << endl;
    return 0;
}
\end{lstlisting}


%--- Przykład zapytania SQL ---
\section*{Przykład zapytania SQL}   
\begin{lstlisting}[language=SQL, caption={Zapytanie SQL wybierające wszystkie rekordy z tabeli 'uzytkownicy'}]
SELECT * FROM uzytkownicy;  
\end{lstlisting}

%===============================================================================================
%--- Przykład tabel ---
\section*{Przykład tabeli}  

\begin{table}[ht!]
\centering
\caption{Przykładowa tabela z trzema kolumnami sformatowana zgodnie z zasadami}
\label{tab:przyklad}
\begin{tabular}{lll}
\toprule
Nazwa elementu & Opis funkcjonalności & Wartość [j.] \\  
\midrule
Element A      & Opis szczegółowy A    & 10 \\
Element B      & Opis szczegółowy B    & 20 \\ 
Element C      & Opis szczegółowy C    & 30 \\
\bottomrule 
\end{tabular}
\sourcetab{opracowanie własne}
\end{table}

%===============================================================================================
%--- Przykład obrazu ---
\begin{figure}[ht]
    \centering
    \includegraphics[width=0.7\textwidth]{rysunki/1.png} 
    \caption{Dodatkowy schemat systemu (import z pliku PNG)}
    \label{fig:schemat_png} % Klucz dla grafiki PNG
\end{figure}

\noindent Na rysunku \ref{fig:schemat_png} zaprezentowano strukturę systemu w ujęciu ogólnym.

%===============================================================================================
%przykład odwołania do bibliografii
\section*{Przykład odwołania do bibliografii}
W literaturze przedmiotu \parencite{nowak2020} omówiono zagadnienia związane z...

%===============================================================================================
%przykład owodłania do netografii
\section*{Przykład odwołania do netografii}
W Internecie dostępne są zasoby dotyczące... \parencite{latex_project}
%===============================================================================================