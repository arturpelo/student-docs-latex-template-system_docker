\chapter{Temat rozdziału}
Tutaj treść rozdziału 1. Możesz dopisać swoje podrozdziały i sekcje tutaj. 
Pamiętaj, aby nie zostawiać pustej linii bezpośrednio przed otwarciem środowiska tabeli, jeśli chcesz, aby była ona bliżej tekstu.

\section{Podrozdział albo sekcja }
Przykładowy tekst, który prowadzi bezpośrednio do prezentacji danych. Warto wspomnieć o literaturze przedmiotu \parencite{kowalski2026}.

\begin{table}[ht!]
\centering
\caption{Przykładowa tabela z trzema kolumnami sformatowana zgodnie z zasadami}
\label{tab:przyklad}
\begin{tabular}{llr}
\toprule
Nazwa elementu & Opis funkcjonalności & Wartość [j.] \\  
\midrule
Element A      & Opis szczegółowy A    & 10 \\
Element B      & Opis szczegółowy B    & 20 \\ 
Element C      & Opis szczegółowy C    & 30 \\
\bottomrule 
\end{tabular}
\sourcetab{opracowanie własne}
\end{table}

Kolejny akapit tekstu zaczyna się tutaj. 