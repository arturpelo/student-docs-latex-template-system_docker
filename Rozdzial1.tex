\chapter{Cel i zakres pracy}

\section{Cel pracy}
Celem niniejszej pracy inżynierskiej jest zaprojektowanie i wykonanie rozwiązania
informatycznego wspierającego przygotowanie dokumentów w~\LaTeX{} oraz automatyzującego
proces kompilacji w~ujednoliconym środowisku uruchomieniowym. Projektowy charakter pracy
obejmuje analizę wymagań, opracowanie architektury, implementację oraz weryfikację poprawności
działania przygotowanego rozwiązania.

\section{Cele szczegółowe}
W~ramach realizacji celu głównego wyznaczono następujące cele szczegółowe:
\begin{itemize}
    \item analiza problemu i zdefiniowanie wymagań funkcjonalnych i niefunkcjonalnych,
    \item zaprojektowanie struktury systemu oraz procesu kompilacji dokumentów,
    \item implementacja rozwiązania wraz z~mechanizmami automatyzacji uruchomienia,
    \item przygotowanie scenariuszy weryfikacyjnych i ocena rezultatów.
\end{itemize}

Zgodnie z~ujęciem przedstawionym w~literaturze, nowoczesne systemy informatyczne powinny
być projektowane z~myślą o~powtarzalności procesów oraz możliwości automatyzacji,
co bezpośrednio wpływa na jakość i niezawodność wytworzonych rezultatów
\parencite{kowalski2026}.

\section{Zakres pracy}
Zakres pracy obejmuje:
\begin{itemize}
    \item opracowanie koncepcji systemu do tworzenia dokumentów oraz jego architektury,
    \item dobór technologii wspierających kompilację w~kontenerach Docker,
    \item implementację skryptów uruchomieniowych i konfiguracji projektu,
    \item weryfikację poprawności generowania dokumentu wynikowego.
\end{itemize}
Poza zakresem pracy pozostają: rozbudowane badania wydajnościowe, tworzenie alternatywnych
silników składu oraz pełna integracja z~zewnętrznymi systemami CI/CD.

\section{Struktura pracy}
Praca została podzielona na pięć rozdziałów. Rozdział pierwszy przedstawia cel oraz zakres.
Rozdział drugi opisuje problematykę projektu i uzasadnia wybór tematu. Rozdział trzeci
zawiera analizę wymagań oraz projekt systemu. Rozdział czwarty prezentuje implementację
i~sposób uruchamiania rozwiązania. Rozdział piąty obejmuje testy i ocenę rezultatów.
