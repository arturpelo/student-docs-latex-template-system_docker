\chapter{Implementacja i uruchomienie}
Rozdział opisuje realizację zaprojektowanego rozwiązania oraz sposób jego uruchomienia
z~punktu widzenia użytkownika końcowego. Przedstawiono środowisko wykonawcze, strukturę
plików i podstawowe kroki pracy z~systemem.

\section{Środowisko i narzędzia}
Implementacja opiera się na \LaTeX{} oraz narzędziu \texttt{latexmk} uruchamianym
w~kontenerze Docker. Takie podejście eliminuje konieczność instalacji pełnego pakietu
\LaTeX{} na komputerze użytkownika oraz zapewnia spójność wyników.

\section{Struktura projektu}
Projekt zawiera plik główny dokumentu, rozdziały tematyczne, pliki bibliografii
oraz katalog z~zasobami graficznymi. Taka organizacja ułatwia zarządzanie treścią
i~rozwój dokumentu w~kolejnych iteracjach.

\section{Proces uruchomienia}
Proces uruchomienia składa się z~następujących kroków:
\begin{itemize}
	\item przygotowanie treści w~plikach \LaTeX{},
	\item uruchomienie skryptu automatyzującego kompilację,
	\item wygenerowanie pliku PDF w~katalogu głównym projektu,
	\item opcjonalne czyszczenie plików tymczasowych po zakończeniu pracy.
\end{itemize}

\section{Obsługa błędów}
W~przypadku problemów z~kompilacją użytkownik otrzymuje czytelny komunikat.
Najczęstszymi przyczynami są: brak uruchomionego Dockera, błędy składni w~plikach
\LaTeX{} lub brak wymaganych zasobów graficznych. Dzięki temu możliwe jest szybkie
zidentyfikowanie problemu i jego korekta.
