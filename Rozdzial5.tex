\chapter{Testy i weryfikacja}
Rozdział przedstawia sposób sprawdzenia poprawności działania rozwiązania oraz ocenę
uzyskanych rezultatów. Testy skupiają się na weryfikacji kluczowych funkcji i jakości
procesu kompilacji.

\section{Strategia testów}
Przyjęto podejście oparte na scenariuszach użycia. Każdy scenariusz odpowiada
konkretnemu wymaganiu funkcjonalnemu lub niefunkcjonalnemu, co pozwala jednoznacznie
ocenić, czy wymaganie zostało spełnione.

\section{Scenariusze weryfikacyjne}
Zastosowano między innymi następujące scenariusze:
\begin{itemize}
	\item kompilacja dokumentu bez błędów i wygenerowanie pliku PDF,
	\item ponowna kompilacja z~uwzględnieniem spisu treści i elementów pomocniczych,
	\item obsługa sytuacji braku uruchomionego Dockera,
	\item reakcja systemu na błędny zapis w~pliku \LaTeX{}.
\end{itemize}

\section{Wyniki testów}
Wyniki potwierdziły poprawność działania procesu kompilacji oraz czytelność komunikatów
z~punktu widzenia użytkownika. Zidentyfikowano typowe źródła błędów związane z~treścią
\LaTeX{} oraz dostępnością środowiska kontenerowego, co uwzględniono w~instrukcjach
uruchomieniowych.

\section{Ograniczenia i możliwości rozwoju}
Testy nie obejmowały pełnych badań wydajnościowych ani integracji z~systemami ciągłej
integracji. W~przyszłości możliwe jest rozszerzenie rozwiązania o~automatyczną walidację
szablonu, raporty z~kompilacji oraz konfigurację zależną od profilu użytkownika.
